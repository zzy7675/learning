\documentclass{article}
\usepackage{listings}
\usepackage{xcolor}
\usepackage{environ} % 用于定义新环境

% 定义代码显示的样式
\lstdefinestyle{mystyle}{
    xleftmargin=2em,
    backgroundcolor=\color{white},    % 代码背景色
    commentstyle=\itshape\color{gray},% 注释的样式
    keywordstyle=\bfseries\color{blue},% 关键字的样式
    breakatwhitespace=false,           % 不在空格处换行
    breaklines=true,                 % 自动换行
    captionpos=b,                    % 标题位置:底部
    keepspaces=true,                 % 保持空格
    numbers=left,                    % 在左侧显示行号
    numbersep=14pt,                   % 行号与代码之间的距离
    showspaces=false,                % 不显示空格
    showstringspaces=false,          % 不显示字符串中的空格
    showtabs=false,                  % 不显示制表符
    tabsize=4                        % 制表符等于两个空格
}

\lstset{style=mystyle}

\begin{document}

\section*{Exercise 1.6}

\begin{lstlisting}[language=C++]
std::cout << "The sum of " << v1;
<< " and " << v2;
<< " is " << v1 + v2 << std::endl;
\end{lstlisting}

\begin{flushleft}
Line 2 and Line 3 do not have a left operand on the left of the first \lstinline[language=C++]{<<} operator. In order to fix it,
we can put \lstinline[language=C++]{std::cout} operand at the start of Line 2 and Line 3 as follows:
\end{flushleft}

\begin{lstlisting}[language=C++]
std::cout << "The sum of " << v1;
std::cout << " and " << v2;
std::cout << " is " << v1 + v2 << std::endl;
\end{lstlisting}

\end{document}

