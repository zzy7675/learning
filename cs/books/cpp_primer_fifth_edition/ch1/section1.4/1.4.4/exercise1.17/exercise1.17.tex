\documentclass{article}
\usepackage{listings}
\usepackage{xcolor}

% 定义代码显示的样式
\lstdefinestyle{mystyle}{
    xleftmargin=2em,
    backgroundcolor=\color{white},    % 代码背景色
    commentstyle=\itshape\color{gray},% 注释的样式
    keywordstyle=\bfseries\color{blue},% 关键字的样式
    breakatwhitespace=false,           % 不在空格处换行
    breaklines=true,                 % 自动换行
    captionpos=b,                    % 标题位置:底部
    keepspaces=true,                 % 保持空格
    numbers=left,                    % 在左侧显示行号
    numbersep=14pt,                   % 行号与代码之间的距离
    showspaces=false,                % 不显示空格
    showstringspaces=false,          % 不显示字符串中的空格
    showtabs=false,                  % 不显示制表符
    tabsize=4                        % 制表符等于两个空格
}

\lstset{style=mystyle}

\begin{document}

\section*{Exercise 1.17}

\begin{flushleft}
If all input numbers are equal, the \lstinline[language=C++]|else| branch in
the \lstinline[language=C++]|while| loop is never executed. After breaking
from the \lstinline[language=C++]|while| loop, the program prints out the 
value and the number of times it occurs.

If all input numbers are distinct, the \lstinline[language=C++]|if| branch in
the \lstinline[language=C++]|while| loop is never executed. The program always
executes \lstinline[language=C++]|else| branch to print out each number and the
occurring time 1 for each number.
\end{flushleft}

\end{document}