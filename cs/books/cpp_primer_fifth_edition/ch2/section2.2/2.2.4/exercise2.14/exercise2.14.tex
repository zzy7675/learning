\documentclass{article}
\usepackage{listings}

\begin{document}
    
\section*{Exercise2.14}

\begin{lstlisting}[numbers=left, basicstyle=\ttfamily, xleftmargin=2em]
int i = 100, sum = 0;
for (int i = 0; i != 10; ++i)
sum += i;
std::cout << i << " " << sum << std::endl;
\end{lstlisting}

\begin{flushleft}
values printed: \lstinline[language=C++]|100 45| \linebreak
i is redefined in \lstinline[language=C++]|for| loop, so sum ends up with the sum of 0 to 9, which is 45. At the start of line 4, the i in the \lstinline[language=C++]|for| loop becomes invalid, and we get back to outer scope. Therefore, the program prints out outer scope i 100.
\end{flushleft}

\end{document}
