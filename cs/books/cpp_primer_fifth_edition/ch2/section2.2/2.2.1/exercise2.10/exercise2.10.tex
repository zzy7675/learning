\documentclass{article}
\usepackage{listings}
\usepackage{xcolor}

\begin{document}

\section*{Exercise 2.10}

\begin{lstlisting}[numbers=left, basicstyle=\ttfamily, language=C++, xleftmargin=2em]
std::string global_str; 
int global_int; 
int main() 
{ 
    int local_int; 
    std::string local_str; 
}
\end{lstlisting}

\begin{flushleft}
\lstinline[language=C++]|global_str| is empty string. \\
\lstinline[language=C++]|global_int| is 0. \\
\lstinline[language=C++]|local_int| is uninitialized. \\
\lstinline[language=C++]|local_str| is initialized by \lstinline[language=C++]|string| class. Thus,
it is initialized to empty string. \linebreak

\section*{Summary}

\lstinline[language=C++]|int| a is built-in type, so there is a difference in initial value when we define it outside and inside a function respectively. \linebreak

\lstinline[language=C++]|string|, however, is actually a class defined in std. Therefore, its initialization depends on its class.
\end{flushleft}

\end{document}
