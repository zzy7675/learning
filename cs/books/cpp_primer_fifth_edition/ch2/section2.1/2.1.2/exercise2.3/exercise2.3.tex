\documentclass{article}
\usepackage{listings}
\usepackage{xcolor}

\lstdefinestyle{mystyle}{
    basicstyle=\ttfamily,
    xleftmargin=2em,
    language=C++,
    backgroundcolor=\color{white},      % 代码背景色
    keywordstyle=\color{blue}, % 关键字的样式
    breakatwhitespace=false,            % 不在空格处换行
    breaklines=true,                    % 自动换行
    captionpos=b,                       % 标题位置:底部
    keepspaces=true,                    % 保持空格
    numbers=left,                       % 在左侧显示行号
    numbersep=10pt,                     % 行号与代码之间的距离
    showspaces=false,                   % 不显示空格
    showstringspaces=false,             % 不显示字符串中的空格
    showtabs=false,                     % 不显示制表符
    tabsize=4                           % 制表符等于两个空格
}

\lstset{style=mystyle}

\begin{document}

\section*{Exercise 2.3}

\begin{lstlisting}
unsigned u = 10, u2 = 42;
std::cout << u2 - u << std::endl; 
std::cout << u - u2 << std::endl; 
    
int i = 10, i2 = 42; 
std::cout << i2 - i << std::endl; 
std::cout << i - i2 << std::endl; 
    
std::cout << i - u << std::endl; 
std::cout << u - i << std::endl;
\end{lstlisting}

\begin{flushleft}
Line 2 prints out 32;                                       \\
Line 3 prints out 4294967264 due to wrap-around(overflow);  \\
Line 6 prints out 32;                                       \\
Line 7 prints out -32;                                      \\
Line 9 prints out 0;                                        \\
Line 10 prints out 0;
\end{flushleft}



\end{document}
