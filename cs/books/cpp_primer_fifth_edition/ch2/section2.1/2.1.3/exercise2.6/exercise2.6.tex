\documentclass{article}
\usepackage{listings}
\usepackage{xcolor}

\lstdefinestyle{mystyle}{
    basicstyle=\ttfamily,
    xleftmargin=2em,
    language=C++,
    backgroundcolor=\color{white},      % 代码背景色
    keywordstyle=\color{blue}, % 关键字的样式
    breakatwhitespace=false,            % 不在空格处换行
    breaklines=true,                    % 自动换行
    captionpos=b,                       % 标题位置:底部
    keepspaces=true,                    % 保持空格
    numbers=left,                       % 在左侧显示行号
    numbersep=10pt,                     % 行号与代码之间的距离
    showspaces=false,                   % 不显示空格
    showstringspaces=false,             % 不显示字符串中的空格
    showtabs=false,                     % 不显示制表符
    tabsize=4                           % 制表符等于两个空格
}

\lstset{style=mystyle}

\begin{document}

\section*{Exercise 2.6}

\begin{lstlisting}
int month = 9, day = 7; 
int month = 09, day = 07;
\end{lstlisting}

\begin{flushleft}
9 and 7 are decimal integer literals in Line 1. Line 1 can be compiled. \\

There are 0 prefix for both literals in Line 2, meaning that they are octal integer literals.
However, 09 exceeds octal integer expression in Line 2. Thus, Line 2 cannot be compiled.
\end{flushleft}



\end{document}
